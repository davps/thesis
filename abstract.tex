% $Log: abstract.tex,v $
% Revision 1.1  93/05/14  14:56:25  starflt
% Initial revision
% 
% Revision 1.1  90/05/04  10:41:01  lwvanels
% Initial revision
% 
%
%% The text of your abstract and nothing else (other than comments) goes here.
%% It will be single-spaced and the rest of the text that is supposed to go on
%% the abstract page will be generated by the abstractpage environment.  This
%% file should be \input (not \include 'd) from cover.tex.
%\subsection*{Resumen}

La norma internacional IEC 61850 es una de las m�s importantes 
en la industria el�ctrica. La misma se aplica principalmente a los
sistemas y redes de comunicaci�n en subestaciones de potencia, con una
penetraci�n creciente en distintas �reas el�ctricas, tales como centrales
hidroel�ctricas. Esta norma especifica un
soporte para la interoperabilidad sustentable entre los dispositivos utilizados
en este contexto (IEDs). La norma IEC 61850 est� dise�ada utilizando una pila
de protocolos de comunicaci�n basados en Ethernet, y define el proceso de
configuraci�n de los dispositivos a trav�s del  Lenguaje de Configuraci�n de
los dispositivos basado en XML el paradigma orientado a objetos, proveyendo un
modelo normalizado para la representaci�n de 
toda la informaci�n del sistema (Nodos L�gicos). Utiliza, adem�s,
t�cnicas avanzadas de comunicaci�n 
con interfaces abstractas de alto nivel que simplifican el dise�o,  
para abordar la gesti�n de datos y simplificar la
integraci�n de aplicaciones.               


Dado que la Central Hidroel�ctrica Itaipu se encuentra en pleno
proceso de actualizaci�n tecnol�gica, 
los estudios referentes a la modernizaci�n de la central 
son de suma importancia. Este trabajo presenta 
el dise�o del modelo IEC 61850 del sistema de regulaci�n 
de velocidad de las unidades generadoras de la Itaipu, 
identificando el proceso de ingenier�a IEC 61850 adecuado a
la capacidad de implementaci�n de la norma por parte de los fabricantes
de IEDs, y buscando la armon�a con las directrices de 
actualizaci�n tecnol�gica de la Itaipu.


 
