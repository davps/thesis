\section{Proceso de ingenier�a totalmente basado en SCL}

%El uso adecuado del \gls{SCL} durante el 
%proceso en ingenier�a trae grandes ventajas 

El flujo de trabajo a ser adoptado durante el proceso
de ingenier�a IEC 61850 de una subestaci�n 
permite dise�ar el sistema 
acorde a lo indicado en el 
apartado IEC 61850--7--1 \cite[cl. 6]{IEC61850-7-1:2003},
\cite[cl. 7.2]{IEC61850-7-1:2003},  
\cite[cl. 8.2]{IEC61850-7-1:2003}.
Todo el dise�o del sistema se puede describir mediante el \gls{SCL}, 
siguiendo los pasos indicados en el apartado 
IEC 61850--6 \cite[cl. 5]{IEC61850-6:2004},
siempre considerando los requisitos 
que exige el dise�o, seg�n las definiciones 
del apartado IEC 61850--4 \cite[cl. 5]{IEC61850-4:2002}.


El \gls{SCL} es  
la parte m�s importante de la 
serie IEC 61850 \cite{Schwarz:2007}, 
pues todo el proceso de ingenier�a 
puede apoyarse en este lenguaje. Todas 
las dem�s partes de la norma (correspondientes 
al dise�o de sistemas), pueden ser expresadas a trav�s 
del \gls{SCL}. 


Durante el transcurrir de este cap�tulo 
se demuestra la viabilidad t�cnica de la 
utilizaci�n del \gls{SCL} durante todo el proceso de 
ingenier�a IEC 61850.


Como primer paso para dicha demostraci�n,
es imprescindible identificar 
el modelo de informaci�n
del sistema que puede ser 
dise�ado mediante \gls{SCL} y  
sus respectivas reglas de
utilizaci�n.
Aplicando 
ingenier�a inversa a 
los XSDs \cite{Hypermodel:2010} de 
IEC 61850-6 \cite{IEC61850-6:2004} es posible 
visualizar con mayor facilidad 
los \textbf{SCL XML Shemas} a trav�s del \gls{UML-es}. 

A medida que en este cap�tulo se van identificando 
las posibilidades 
que ofrece el \gls{SCL}, el autor concibe
el enfoque m�s adecuado 
para resolver el problema planteado.

