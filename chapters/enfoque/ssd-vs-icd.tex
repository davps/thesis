\section{Especificaci�n de nodos l�gicos en variantes SSD vs. especificaci�n
con variantes ICD}
	\label{sec:ENFOQUE-ssd-vs-icd}
	
	
Seg�n las definiciones de la norma 
IEC 61850--6 \cite{IEC61850-6:2004}, 
la variante SSD puede ser utilizada 
para definir el diagrama unifilar y 
los nodos l�gicos requeridos en dicho
diagrama. 

Seg�n el XML Shema de la norma \cite{IEC61850-6:2004} que define la estructura del SLC
no es obligatorio definir correctamente 
el atributo \emph{iedName} del 
elemento \emph{LNode} en la variante SSD, debido a esto, 
la especificaci�n de nodos l�gicos de IEDs  (productos)
con la variante SSD puede 
inducir a errores, dado que en un SSD (aunque est� en 
plena conformidad con la norma IEC 61850) 
no valida en forma autom�tica 
la definici�n de ubicaciones de nodos l�gicos 
en los respectivos IEDs y tampoco se provee 
la agrupaci�n en Dispositivos L�gicos sin
antes haber creado el SSD (generalmente
los sistemas se construyen con cientos de instancias 
de nodos l�gicos, por lo que deber�a haber 
una asistencia por parte del software de modo
a minimizar los errores humanos). 
Si las herramientas 
de ingenier�a permitieran forzar la identificaci�n 
de los nodos l�gicos en sus respectivos IEDs
y posibilitaran la visualizaci�n en modo gr�fico 
o con tablas de las agrupaciones de LNodes realizadas 
en archivos SSD 
este problema ser�a resuelto, 
pero como no lo hacen \cite{PTI:SESEP2010}, 
existe una alta probabilidad de que 
algunos nodos l�gicos
no est�n bien ubicados, o directamente  
no est�n ubicados a 
ning�n IED en variantes SSD 
dise�adas para especificaciones t�cnicas 
de IEDs (esto se soluciona utilizando 
gr�ficos hechos en programas tales como AutoCAD, 
pero en este trabajo se busca
un enfoque a trav�s del cual se pueda obtener
un dise�o IEC 61850 totalmente basado en SCL). 

Es por ello que el autor de este trabajo 
ha analizado otras alternativas que asistan  
las especificaciones simplificadas 
y dise�os del modelo de sistema IEC 61850 realizado.
La especificaci�n de nodos l�gicos con la
variante ICD que posea una profundidad de 
descripci�n adecuada y bien definida resuelve 
todos los problemas que se podr�an ocasionar 
al especificar con variantes SSD como producto 
final de la especificaci�n.

El SSD creado al final del trabajo tiene en cuenta
los siguientes v�nculos XML:

\lstinputlisting[label=cod:slc-ssd-ln-constraints,
caption=Reglas definidas en IEC 61850-6 para la construcci�n de nodos l�gicos en
la variante SSD] {chapters/enfoque/source/scl/ssd-ln-constraint.xsd} 

por lo que el resultado del SSD es muy aproximado 
al SCD, dado que ya se realiza el mapeo de los 
\textbf{LN} de ICDs a \textbf{LNodes} del SSD. 

En la siguiente secci�n, el autor describe este
tipo de variante ICD mencionado anteriormente. 