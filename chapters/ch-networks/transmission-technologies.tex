\section{Transmission technologies}

Types of transmission technology:
\begin{itemize}
  \item Broadcast links.
  \item Point-to-point links.
\end{itemize}

Broadcast networks have a single communication channel 
that is shared by all the machines on the network. 
Short messages, called packets in certain contexts, 
sent by any machine are received by all the others. An 
address field within the packet specifies the intended 
recipient. Upon receiving a packet, a machine checks the 
address field. If the packet is intended for the receiving 
machine, that machine processes the packet; if the 
packet is intended for some other machine, 
it is just ignored. Some broadcast systems also 
support transmission to a subset of the machines, 
something known as multicasting \cite{Tanembaum:2003cn}. 

In contrast, point-to-point networks, sometimes 
called unicasting, consist of
many connections between individual pairs of machines. To go 
from the source to the destination, a packet on this 
type of network may have to first visit one or more 
intermediate machines. Often multiple routes, of 
different lengths, are possible, so finding good 
ones is important in point-to-point 
networks \cite{Tanembaum:2003cn}.

