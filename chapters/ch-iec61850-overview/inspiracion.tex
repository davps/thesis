puedo decir de que se eligio a la tecnologia xml 
como base para la parte 6 por ser muy 
utilizado en la industria del software hoy en dia.
Existen muchas tecnologias montadas sobre, por y 
para xml. 

Por dar unos ejemplos, cito el xml data binding jaxb de java,
o la persistencia donde se configura con xml, 
tambien C++ tiene librerias que hacen el 
xml data binding.
Hay base de datos que te entregan un xml y 
se usa el data binding tambien para convertirlos 
a objetos.
Los esquemas ayudan a crear los objetos 
en conformidad 
con las especificaciones. 
Tambien xml se puede procesar desde 
cualquier lenguaje de programacion 
o cualquier sistema operativo. esa 
es otra ventaja por la cual se eligio 
esa plataforma. Hay toda una infraestructura 
atras de esto y es bueno para montar 
herramientas de ingenieria muy 
sofisticadas, de forma facil.









--------------------------------------------
la filosofia actual de la norma es que facilita 
la interoperabilidad enfocandose en crear 
una especificacion en xml para las 
diversas configuraciones y puesta a 
punto de los ieds. 
Se podria crear un nuevo frente de 
ataque para mejorar el esfuerzo 
hacia la interoperabilidad dando 
definiciones bien especificadas de 
una api para cada lenguaje de programacion.
Esto no solo permitiria la facil configuracion
de dispositivos, sino tambien abre las 
puertas para la construccion de una nueva gama
de aplicaciones de software para el ambito electrico,
permitiendo la interoperabilidad a nivel 
de lenguaje de programacion. Asi como la 
especificaciones de java por ejemplo, y 
sus diversos implementadores.


The scope of SCL as defined in this standard is ``clearly restricted'' to these purposes:
1)   SAS functional specification (point  a) above),
2)   IED capability description (points  b)and  c) above), and
3)   SA system description (points  d) and  e) above)




switchyard???  q es?