\section{Research Methodologies and Techniques}
	\todo[inline]{Falta traducir al ingl\'es toda esta secci\'on}
	\subsection{Investigaci\'on bibliogr\'afica} 
		Incluye la investigaci\'on y estudio de la norma IEC 61850, en especial las
		definiciones de modelos de informaci\'on definidos en la parte IEC
		61850-7-4-10 \emph{Basic communication structure for substation and feeder
		equipment – Compatible logical node classes and data classes - Hydro Power
		Plants}, los servicios de intercambio de informaci\'on para diferentes
		funciones (por ejemplo, control, reporte, \emph{getters} y \emph{setters})
		definidos en el apartado IEC 61850-7-2 \emph{Basic communication structure
		for substation and feeder equipment – Abstract communication service
		interface (ACSI)}, la implementaci\'on de dichos servicios de comunicaci\'on
		a trav\'es de protocolos especificados en la parte IEC 61850-8-1
		\emph{Specific Communication Service Mapping (SCSM) – Mappings to MMS (ISO
		9506-1 and ISO 9506-2) and to ISO/IEC 8802-3} para las necesidades de la
		unidad generadora, y otros documentos, normas, y tecnolog\'ias relacionadas.
	\subsection{Investigaci\'on de campo}
		Incluye el levantamiento de informaciones del generador de Itaip\'u, la
		identificaci\'on de las funciones de automatizaci\'on inherentes regulador de
		velocidad de la unidad generadora y otras documentaciones y estad\'isticas
		relacionadas al caso en estudio. Tambi\'en incluye la simulaci\'on en una
		red IEC 61850 utilizando las herramientas de ingenier\'ia disponibles en el
		mercado para tal efecto.
	\subsection{Creaci\'on de los objetos}
		Dise\~no de los nodos l\'ogicos normalizados en la parte IEC 61850-7-4, e
		IEC 61850-7-4-10 mediante UML, implementaciones en XML, Java, C++, o
		mediante las herramientas de ingenier\'ia especializadas en la norma IEC
		61850 disponibles en el mercado.
	\subsection{Identificaci\'on de servicios de informaci\'on}
		Modelado completo de los servicios de comunicaci\'on necesarios para el
		regulador de velocidad de una unidad generadora t\'ipica de Itaipu
		utilizando la parte IEC 61850-7-2.
	\subsection{Propuesta de extensi\'on/complementaci\'on de los nodos l\'ogicos
		e ICDs} Por ejemplo, ZAXN, insuficiente para la alimentaci\'on AC y DC de los
		servicios auxiliares de Itaipu.
	\subsection{Estructuraci\'on de una metodolog\'ia de aplicaci\'on de la norma
		IEC 61850 en la automatizaci\'on de hidroel\'ectricas} Elecci\'on de las
		herramientas de ingenier\'ia disponibles en el mercado y que mejor se adapten
		a la automatizaci\'on de unidades generadoras aplicando la norma IEC 61850 y
		la creaci\'on de procedimientos padronizados utilizando dichas herramientas,
		para satisfacer los requerimientos de ingenier\'ia definidos en el apartado
		IEC 61850-4 subsecci\'on 3, y para identificar las funciones de
		automatizaci\'on que son necesarias para el modelado de los nodos l\'ogicos,
		pero no est\'an contempladas en la norma IEC 61850.
