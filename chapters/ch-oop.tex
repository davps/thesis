%%Este capítulo debe ser el primer tema técnico a tratar,
%%debido a que mi tema se trata al modelado de objetos 
%%y de servicios de comunicación (al final los servicios
%%de comunicación se modelan como objetos via xml)
%%y también la utilización de interfaces da una buena base
%%para entender las capas del modelo OSI: las interfaces
%%que posee cada capa para comunicarse con las otras
%%(servicios) son muy bien compreendidas una vez
%%tratada el tema de interfaces de OOP. Al final, 
%%este tema de interfaces es 
%%abstracción, tratada en ingeniería de software. Por 
%%ello, es un buen punto empezar desde aquí.

\chapter{Object-oriented system construction}

\section{Introduction}

This chapter describes the object-oriented programming (OPP) paradigm 
elements that will be used as the basis for the objects models 
in the IEC 61850 standard. The chapter does not describe 
general OPP principles, only focuses on 
the neccesary principles that will be used 
on the IEC 61850 engineering.
\section{Object Oriented systems}


%\todo[inline]{leer el paper de historia de poo, en la 2da pagina}
%\cite{Wegner:1987}

Many autors formulate precise 
definition of O-O paradigm 
\cite{Rentsch:1982} 
\cite{Pascoe:1986}
\cite{Nygaard:1986}
\cite{Madsen:1988}, 
and the definition of Lastly, Wegner 
\cite{Wegner:1987} has been 
the most widely 
accepted \cite{Capretz:2003}. Wegner defines 
the O-O paradigm in terms of objects, 
classes and inheritance.

\emph{
	``Objects are autonomous entities 
	that respond to messages or operations and share 
	a state. Classes classify objects by their common 
	operations. Inheritance serves to classify classes by 
	their shared behavior. Data abstraction hides the 
	representation of data and the implementation of 
	operations'' 
}\cite{Wegner:1987}. That is: 

object oriented = objects + classes + inheritance

In the next sections theses concepts are explained.

%This paradigm is useful to develop scalable, 
%consistent and reliable software systems 
%organizing the code to create objects. 
The O-O system programming metodology 
defines an approach to code organization 
for objects creation. 
The objects store the data and 
have their own behabiour with a 
particular information grouping 
by common functionality and common 
information structures. 
O-O systems benefits 
to bundle actions together, 
manage a few quantity of variables rather 
than multiples ones, 
organizing the 
common behavior together and 
structure programs in a way that
matches closely 
the real world \cite{Adobe:AS3man2008}. 


\section{Classes}

%%TODO: cite adobe book
%\input{chapters/ch-oop/classes-definitions}
A class is an abstract representation of an object. 
A class stores information about the types of data 
that an object can hold and the behaviors that an 
object can exhibit.

\subsection{Methods}
Methods are functions that are part of a class 
definition. Once an instance of the class is created, 
a method is bound to that instance.

	\subsubsection{Get and set accessor methods}
	Get and set accessor functions, also called getters 
	and setters, allow you to adhere to the programming principles of 
	information hiding and encapsulation while providing an 
	easy-to-use programming interface for the classes that you 
	create. Get and set functions allow you to keep your class 
	properties private to the class, but allow users of your class 
	to access those properties as if they were accessing a 
	class variable instead of calling a class method. 
	The advantage of this approach is that you can avoid 
	having two public-facing functions for each property 
	that allows both read and write access.
	
	\subsubsection{Constructor methods}
	Constructor methods, sometimes simply called constructors, 
	are functions that share the same name as the class in 
	which they are defined. Any code that you include in 
	a constructor method is executed whenever an instance of the 
	class is created with the  new  keyword.

\section{Objects}

The O-O program interaction is realized by 
objects invoking methods. 

\todo[inline]{falta completar}
\section{Diference betwen Classes and Objects}

They are a clear separation between 
class and object, and they are 
the most important concepts of O-O programming 
languages. \cite{Dahl:1970} 
%this separation was fist introduced by Simula 
%to simulate real-world applications. 

A class is a static off-line 
template to generate objects. 
A object is a runtime data 
or a group of datas 
according to the class. The objects are 
independend datas with 
its own behavior, and 
they are classified by the classes 
\todo{ESP:por las clases q la han generado}
which generated.


\section{The objects relationships}

\todo[inline, color=blue!50!red!50]{esto no va en la
seccion de objetos, pues antes necesite explicar 
la secci\'on \ref{sec:Diference-betwen-Classes-and-Objects} 
que trata sobre la diferencia entre objeto y clase. 
Esto es importante como una previa para esta seccion dado 
que las subsecciones a continuacion se presentan como 
clases pero describen las relaciones 
de los objetos. por eso es importante entender bien la 
diferencia primero. Esto es por el lector, para que 
le sea mas facil entender.}


	\subsection{Association}
		\todo{completar - association}
	
	\subsection{Aggregation}
		\todo{completar - aggregation}
		
	\subsection{Composition}
		\todo{completar - composition}
		
	\subsection{Other relationships}
		\todo[inline, color=blue!50!red!50]{por el momento 
		aca solo hay uno, por eso uso una subsubsection, 
		pero es altamente probable que coloque mas cosas
		si es que encuentro que son utilizados en la norma.}
		\subsubsection{Dependency}
			\todo{completar - dependency: la flechita del uml}

\section{Inheritance}

The inheritance can be used as a way to define new classes 
based in more general classes that have been 
defined previously, acquiring their characteristics.  
If a class \emph{CurrentTransformer} directly 
inherits from class \emph{Transformer} we say that 
\emph{Transformer} is the parent 
of \emph{CurrentTransformer} and 
\emph{CurrentTransformer} is the 
child of \emph{Transformer} \cite{Snyder:1986}. 
The UML representation is given in 
Figure \ref{fig:inheritance-fig}.



\begin{figure}
  \includegraphics[width=0.3\textwidth]{chapters/ch-oop/figures/inheritance}
  \caption{
  		Inheritance: \emph{Transformer} is the parent 
		of \emph{CurrentTransformer} and 
		\emph{CurrentTransformer} is the 
		child of \emph{Transformer}
		}
  \label{fig:inheritance-fig}
\end{figure}


\section{Information hidding and encapsulation}

	\subsection{Abstract data types}

Abstraction and information hidding form 
the foundation of all object-oriented 
development \cite{Levy:1984}. 
An abstraction is a simplified description, 
or specification, of a system that emphasizes 
some of the system's details or properties while
suppressing others \cite{Shaw:1984}.
Information hidding, as first promoted by Parnas,
\todo{explicar mas c/mis palabras} 
goes on to suggest that we should decompose 
systems based upon the principle of hidding 
design decisions about our 
abstractions \cite{Parnas:1979} \cite{Grady:1995}.

The abstraction and information hidding 
are very common in electrical equipments and 
mathematical representations of the 
electrical world. The 
models are abstracted 
and is possible to identify the object and 
operations that exist at each level of integrations 
Thus, 
when \todo[inline]{ver si las
palabras estan bien escritas
en este ejemplo} we work 
with phasors to represent a current, which 
leave just the static amplitude and phase
information. The time space are hidden with 
the purpose to manage the information 
at a more hight level, 
thereby skipping the trigonometric calculations 
derived from the time dependence of the sine wave, 
and the information are combined just algebraically, 
simplifying certain kinds of complex 
calculations.  \cite{Grady:1995}

%repetido!
%The abstraction  and information hidding are common 
%in our activities, we abstract the models by 
%identify the object and operations that exist 
%at each level of integration. Thus, when
%\todo[inline]{ver si las palabras estan 
%bien escritas en este ejemplo} 
%working a transformer, we consider the 
%taps, the current on the low and hight side, 
%the transformation relation \cite{Grady:1995}.

The use of abstract data types on a object oriented system 
help to a more precise and at the same time simple on the 
specification taking adventage of the information hidding 
provided by abstract 
data types. \cite{} \todo{aca debo citar mi paper sobre ACSI}





	\section{Intefaces}

%\input{chapters/ch-oop/blah blah blah}

	\section{Encapsulation}

\todo[inline]{Debo colocar algo aca}
	\subsection{Interfaces and implementation}

\todo[inline]{modificar, reelaborar con mis palabras}

Interfaces are based on the distinction between a 
methods interface and its implementation. A method’s interface 
includes all the information necessary to invoke that 
method, including the name of the method, all of its 
parameters, and its return type. A method’s 
implementation includes not only the interface 
information, but also the executable statements 
that carry out the method’s behavior. An 
interface definition contains only method 
interfaces, and any class that implements 
the interface is responsible for defining the 
method implementations. 
%%TODO: falta explicar más, enfocando desde 
%%el punto de vista de ingeniería de software
\cite[pp.~90-105]{Adobe:AS3man2008}

\todo[inline]{agregar aca lo que subraye
y resum� en
la segunda pagina del paper snyder:1986}	

\section{Events}



	\subsection{Event Dispatching}
	
	
	

\section{Exception handling}

%brevemente inspirado en este link
%http://www.allinterview.com/showanswers/59519.html
The exception is an unwanted and 
unexpeceted event which occours 
at runtime, and are caused by an unusual 
situation of the software that  
stop the normal secuence of  
software operations. 

When a exception event are 
dispatched by the system, 
the error could be handled 
by a specific modelled object 
if the software developer anticipated  
it as a posibility. 

The exception could be modelled 
with different handling stategies,     
\cite{MoonStallman:1983}
\cite{Dony:1988}
\cite{DonyC:1990}
\cite{Leavens:1991}
\todo[inline]{buscar mas papers de referencia 
			para colocar aqui por si alguien 
			tenga interes de profundizar en esto
			(buscar en la acm) }
%for example, could be represented 
%by classes of which 
%concrete exceptions would be some
%referenceable instances.  
for example, could be represented 
by classes specifically 
designed for this purpose of which 
concrete exceptions would be some
referenceable instances that 
handles the event of error. 

\begin{comment}
	\todo[inline]{leer la siguiente bibliografia:
	
	[MSW83] David Moon, Richard M. Stallman, and  Daniel Weinreb.  Lisp Machine 
	Manual  (fifth edition). Massachusetts Institute of Technology, Artificial Intelligence Laboratory, Cambridge, Mass., 
	January  1983.
	
	5.4 Exception Handling
	Hierarchies can also be used to classify 
	and organize exceptions in large 
	software systems. I think this was
	first done in the Flavors mechanism 
	of the Lisp Machine [MSW83]. Recent 
	papers on this topic include
	[Don88] [Don90].
	
	Extraido del paper: 
	introduction to the literature on O-O design, 
	programming and languages.
	Gary t. leavens
	}

esta tambien es una buena partida para saber los distintos
tipos de errores que no trato aqui por ser innecesarios
para la norma
http://www.sap-img.com/abap/difference-between-error-and-exception.htm
\end{comment}

\section{Serialization \todo[color=green!40]{61850, parte6, cl6.1, \textparagraph 3}
}
\todo[inline]{este parrafo a continuacion lo explique
con mis palabras, debo buscar algun paper que habla 
con un lenguaje mas tecnico para apoyar mi explicacion
y agregar las referencias correspondientes}
Serialization, or marshalling is the process of persist the object in memory 
or for transmission purpose, by creating a human readable 
archive that containts the object structure and datas 
in a saveable way. A object just exist at runtime, and 
it is serializad to be saved over the time or 
to be alternativatelly retransmited by the network 
to another host.
\section{Object oriented develoment approach}

The object-oriented development should 
follow the steps proposed by Aboott \cite{Abbott:1980}:  

\begin{itemize}
  \item Identify the objects and their attributes.
  \item Identify the operations suffered by and required
		of each object.
  \item Establish the visibility of each object in 
		relation to other objects.
  \item Establish the interface of each object.
  \item Implement each object.  
\end{itemize}


