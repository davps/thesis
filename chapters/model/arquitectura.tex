\section{Arquitectura del sistema de 
monitoreo y control del regulador de velocidad}

La arquitectura del sistema IEC 61850 definido en este
trabajo tiene las siguientes caracter�sticas:

\begin{itemize}
  \item Topolog�a de red: Anillo.
  \item Redundancia: Sistema totalmente duplicado e id�ntico.
  \item Regulaci�n de velocidad: La regulaci�n primaria y la regulaci�n
  secundaria es realizada en IEDs distintos.
  \item Sensores de frecuencia: Se modelaron ICDs b�sicos
  (conteniendo solamente un nodo l�gico por fase, y suponiendo 
  que la conexi�n del secundario del transformador est�n en 
  estrella para simplificar a�n m�s el modelo) 
  de dos \glspl{MU} para obtener
  la frecuencia del generador y la frecuencia del sistema de potencia.
  \item Sistema Hidr�ulico: El sistema contiene 3 IEDs sensores 
  para la parte hidr�ulica.
  \item Tac�metro: Se ha definido un tac�metro en un ICD. Se 
  han agregado varios nodos l�gicos para el ajuste de curvas, 
  en caso que sea necesario.
\end{itemize} 

La arquitectura general referencial puede visualizarse en la figura
\ref{fig:arquitectura-Gral-Referencial1}. Se ha utilizado 
el layout de la arquitectura general referencial de la 
futura Subestaci�n Villa Hayes \cite{Itaipu:6693DE15207E}.

\begin{landscape}
\thispagestyle{empty}
\begin{figure}
\begin{center}
  \includegraphics[width=0.65\linewidth]{chapters/model/figures/arquitecturaGralReferencial.eps}
  \captionsetup{font=scriptsize}
  \caption{Arquitectura del sistema}
  \label{fig:arquitectura-Gral-Referencial1}
\end{center}
\end{figure}
\end{landscape}
