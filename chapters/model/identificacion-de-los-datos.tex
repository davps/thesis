\section{Datos del sistema}

%Uno de los primeros pasos del proceso de ingenier�a 
%IEC 61850 consiste en identificar los datos
%disponibles, y en base a estos datos, determinar 
%los nodos l�gicos correspondientes.

En este trabajo se toma como punto de partida los documentos
\emph{Itaipu, aspectos destacados de su ingenier�a} \cite{Itaipu:195860C8841ER0},
\emph{Governing Diagram} \cite{Itaipu:5215DF71167I1R1}, y 
\emph{Governing Diagram -- Description} \cite{Itaipu:52151071168i1r1}, 
donde ya se detallan todos los datos del regulador
de velocidad. Muchos de estos datos no necesariamente
se env�an por la red. Se ha dise�ado el modelo de  
todos los datos posibles para tener un mayor
grado de virtualizaci�n de los sistemas reales, 
de este modo, la programaci�n de las funciones
de control dentro de los IEDs manejan 
datos sem�nticos que podr�an facilitar 
el dise�o de los bloques de control 
dentro del IED o tambi�n el monitoreo del sistema.

Los nodos l�gicos son presentados en un formato tabla. El dise�o del modelo IEC 61850 
presentado en este trabajo se basa en los modelos de nodos l�gicos presentados por esta norma. En 
otras palabras, la combinaci�n seleccionada de nodos l�gicos (modelos de informaci�n definidos por la norma y presentados como
\textbf{DataTypeTemplates} en este trabajo) constituye el dise�o del sistema (modelo de informaci�n del sistema, presentado como
instancias de nodos l�gicos). En resumen, en este cap�tulo se utilizan:
  
\begin{itemize}
	\item Tablas: Enumeran las instancias de un determinado nodo l�gico. Las descripciones incluyen los datos de las unidades generadoras
	de Itaipu.
	\item DataTypeTemplates: Plantillas \textbf{LNType}, del elemento \textbf{DataTypeTemplates}. Estos modelos son extraidos del apartado 7-4-10 
	de la norma IEC 61850 y contiene las descripciones proveidas por esta norma, sin incluir los datos de las unidades generadoras de Itaipu. Cuando 
	el lector desee conocer la aplicaci�n de estas plantillas de nodos l�gicos deber� observar la tabla de instancias del nodo l�gico. 
	\item Descripci�n en \GLS{SCL}: Para ofrecer una descripci�n formal este trabajo tambi�n presenta el dise�o de los modelos 
	de plantillas e instancias usando \GLS{SCL}, que es el lenguaje definido por la norma IEC 61850.
\end{itemize}

