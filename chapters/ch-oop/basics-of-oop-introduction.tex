\subsection{Introduction to object-oriented programming}

%TODO: citar
%esta parte fue obtenida del libro de adobe, del cap 5. 
Object-oriented programming (OOP) is a way of organizing the code 
in a program by grouping it into objects-individual elements that include 
information (data values) and functionality. Using an 
object-oriented approach to organizing a program allows 
you to group particular pieces of information (for 
example, a automation function or a current value) together with 
common functionality or actions associated with that 
information (such as ``switchgear actuation'' or 
``voltage measurement''). These items are combined into a single 
item, an object (for example, an 
\todo[size=\tiny]{cambiar por un ejemplo el\'ectrico}
``Album'' or ``MusicTrack''). Being 
able to bundle these values and functions together provides several 
benefits, including only needing to keep track of a single 
variable rather than multiple ones, organizing related 
functionality together, and being able to structure 
programs in ways that more closely match the real world.  

