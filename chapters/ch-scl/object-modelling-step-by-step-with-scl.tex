\section{A practical object and 
communication services modelling example with SCL}


\subsection{Namespaces}
	This element must appear first among all sub-elements, 
	and provide information about the \gls{SCL}['s] \gls{XSD}  
	to be assigned in the SCL engineering process.  
	
	\lstinputlisting[label=code:namespacesXML,
	caption=IEC 61850 SCL Namespaces
	]{chapters/ch-scl/source/scl/header/namespaces.xml}





\subsection{Header}
	The header identify a SCL file, gives information about 
	the version of the SCL file, and the tool used for the 
	SCL engineering.
	\lstinputlisting[label=code:headerXML,
	caption= SCL header
	]{chapters/ch-scl/source/scl/header/header.xml}




%%%Substation
\subsection{Substation}
	The substation xml node group the primary power system topology
	and their respective functions.
	The substation model are described in the section
	\ref{sec:ch-scl--SCL-object-model}.
	
	\lstinputlisting[label=code:substationXML,
	caption= SCL Substation node
	]{chapters/ch-scl/source/scl/substation/substation.xml}




%%%VoltageLevel
\subsection{Voltage Level}
	The VoltageLevel tag allow the agrupation 
	of various levels of switchyard equipments in the same 
	potential. %under a this object.
	In the listing \ref{code:voltageLevelD1XML} 
	are defined a first level of voltage called ``\emph{D1}'':
	
	\lstinputlisting[label=code:voltageLevelD1XML,
	caption= SCL Substation voltage level D1
	]{chapters/ch-scl/source/scl/substation/voltageLevel/voltageLevelD1.xml}
	
	and the second level of voltage of this example is provided
	in the listing \ref{code:voltageLevelE1XML}:
	\lstinputlisting[label=code:voltageLevelE1XML,
	caption= SCL Substation voltage level E1
	]{chapters/ch-scl/source/scl/substation/voltageLevel/voltageLevelE1.xml}
	
	grouping theses voltage levels into the substation:
	\lstinputlisting[label=code:voltageLevelXML,
	caption= SCL Substation voltage levels
	]{chapters/ch-scl/source/scl/substation/voltageLevel/voltageLevel.xml}
	
	and finally, assigning determined voltages to the 
	diferents voltage levels:
	\lstinputlisting[label=code:voltageLevel-and-voltage-XML,
	caption= SCL Substation voltage levels containing 
	their respective voltage
	information]{chapters/ch-scl/source/scl/substation/voltageLevel/voltageLevel-and-voltage.xml}


\subsection{Bay}
	A bay object modelling is analogue to the voltage level. 
	A voltage level can contain several bays.
	The bay object are represented here:
	\lstinputlisting[label=code:bay-XML,
	caption= Bay ]{chapters/ch-scl/source/scl/substation/voltageLevel/bay/bayQ1.xml}
	
	And here several bays of a unique voltage level.
	\lstinputlisting[label=code:bay-XML,
	caption= Bay as child of the voltage level D1 
	]{chapters/ch-scl/source/scl/substation/voltageLevel/bay/bayQ1Q2Q3.xml}
	
	and 4 bays in a substation. The references to 
	theses bays are:
	\begin{itemize}
	  \item S12/D1/Q1
	  \item S12/E1/Q1
	  \item S12/E1/Q2
	  \item S12/E1/Q3
	\end{itemize}  
	Note that the bay S12/D1/Q1 is different that the 
	bay S12/E1/Q1. Both of them have the name Q1, but 
	they have different paths. Then, theses references 
	do not point to the same object.
	\lstinputlisting[label=code:bay-XML,
	caption= 2 Bays in the Substation context. 
	]{chapters/ch-scl/source/scl/substation/voltageLevel/bay/bayQ1Q2Q3-2VoltageLevels.xml}

\subsection{Conducting Equipment}
The conducting equipment are \todo{completar\ldots}









\subsection{Complete SSD}
The complete SSD are provided here:
%scl-Example-T1-1-SSD-complete.xml
%\lstinputlisting[label=code:TCTR_v1java,
%caption=Example
%of SSD]{chapters/ch-scl/source/scl/scl-Example-T1-1-SSD-complete.xml}





