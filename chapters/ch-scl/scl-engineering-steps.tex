\section{SCL engineering specialisations} 
\label{sec:SCL-engineering-parts}
The SCL based communication and network description 
is an important part of the \gls{SAS} engineering, 
and involves multidisciplinary knowledge areas. 
For this reason, the division of labour 
for SCL engineering is necessary.  
It may start either with the 
SAS functional specification, 
\gls{IED} capability description
or with the \gls{SAS} description. 
Theses steps are explained 
in the following subsections: 

\subsection{SAS functional specification}
The functional specification input 
to \gls{SAS} engineering 
consist of system specification in terms of 
single line diagram, allocation of the \glspl{LN} 
and equipmets of the single line diagram.

\subsection{IED capability description}
Also called \emph{IED pre-engineering}. 
In this step are described the IED capabilityes, for 
example, a IED supporting the double busbar line feeder function.

\subsection{SAS description}
\label{sec:SAS-description-scl-engineering}
This is part of the \gls{SAS} engineering, where the 
complete process configuration takes place: All IED 
are bounded to individual process functions and primary 
equipments, enhanced by access point connections and 
the access path in subnetworks for the clients.

A complete \gls{SAS} description provides 
all predefined associations and client-server connections 
(the \gls{IED} cannot built it automatically).

\begin{landscape}
	\begin{figure}
	  %\includegraphics[width=1.0\textwidth]{chapters/ch-scl/figures/SCL-development-process}
	  \includegraphics[width=1.0\linewidth]{chapters/ch-scl/figures/SCL-development-process}
	  \caption{SCL engineering process}
	  \label{fig:SCL-development-process}
	\end{figure}
\end{landscape}

