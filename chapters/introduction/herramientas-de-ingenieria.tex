\section{Herramientas de ingenier�a}

Como apoyo fundamental para la ejecuci�n de estas tareas, la norma IEC 61850
contempla y define las llamadas herramientas de ingenier�a, que son programas
altamente especializados concebidos para elaborar los archivos necesarios para
especificar y configurar el sistema de automatizaci�n de una subestaci�n
el�ctrica que incorpore a la norma IEC 61850 como patr�n de comunicaciones [2].
Estas herramientas deber�an ofrecer una amplia gama de funcionalidades, como,
por ejemplo, la posibilidad de configurar dispositivos de marcas diferentes que
integren un mismo sistema, con base en las caracter�sticas de interoperabilidad
perseguidas por la norma \cite{PTI:SESEP2010}.


De acuerdo a esta norma, las herramientas de ingenier�a necesarias para los
proyectos realizados en conformidad con la misma deben posibilitar la creaci�n
y documentaci�n de los procesos de ingenier�a, tales como: gerenciamiento del
proyecto, parametrizaci�n de dispositivos y documentaci�n del sistema de
automatizaci�n de subestaciones por medio de la utilizaci�n del \gls{SCL}.         

