\section{Partes de la norma IEC 61850 utilizadas para modelar la informaci�n de centrales hidro--el�ctricas}

De entre las partes mencionadas en la secci�n anterior destacamos las m�s importantes para este trabajo de modelado de la informaci�n 
del sistema de regulaci�n de una unidad generadora t�pica de Itaipu:

\begin{itemize}
  \item IEC 61850--7--4:2003	\cite{IEC61850-7-4:2003}:  
	Basic communication structure -- 
	Compatible logical node classes and data classes  (IS Ed1:2003--05).
  \item IEC 61850--7--410:2007	\cite{IEC61850-7-410:2007}:  
	Hydroelectric power plants -- 
	Communication for monitoring and control  (IS Ed1:2007--08).
\end{itemize}

La parte 7--4--10 define formalmente los nodos l�gicos para centrales hidroel�ctricas, mientras que la parte 7--4 
define los nodos l�gicos m�s generales, utilizados tanto en hidroel�ctricas, subestaciones, u otra �rea del sistema el�ctrico. 

Este apartado tiene relaci�n con la utilizaci�n de los nodos l�gicos en hidroel�ctricas, pero 
durante la realizaci�n de este trabajo este documento a�n se encontraba en proceso de redacci�n, 
y no se ha tenido acceso al borrador durante la elaboraci�n del presente trabajo.
\begin{itemize}
  \item  IEC 61850--7--510:2009	\cite{IEC61850-7-510:2009}:  
	Use of logical nodes to model 
	functions of a hydro power plant  (DC 2009--12).
\end{itemize}

